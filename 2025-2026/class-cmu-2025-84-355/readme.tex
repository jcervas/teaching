% Options for packages loaded elsewhere
\PassOptionsToPackage{unicode}{hyperref}
\PassOptionsToPackage{hyphens}{url}
%
\documentclass[
]{article}
\usepackage{amsmath,amssymb}
\usepackage{lmodern}
\usepackage{iftex}
\ifPDFTeX
  \usepackage[T1]{fontenc}
  \usepackage[utf8]{inputenc}
  \usepackage{textcomp} % provide euro and other symbols
\else % if luatex or xetex
  \usepackage{unicode-math}
  \defaultfontfeatures{Scale=MatchLowercase}
  \defaultfontfeatures[\rmfamily]{Ligatures=TeX,Scale=1}
\fi
% Use upquote if available, for straight quotes in verbatim environments
\IfFileExists{upquote.sty}{\usepackage{upquote}}{}
\IfFileExists{microtype.sty}{% use microtype if available
  \usepackage[]{microtype}
  \UseMicrotypeSet[protrusion]{basicmath} % disable protrusion for tt fonts
}{}
\makeatletter
\@ifundefined{KOMAClassName}{% if non-KOMA class
  \IfFileExists{parskip.sty}{%
    \usepackage{parskip}
  }{% else
    \setlength{\parindent}{0pt}
    \setlength{\parskip}{6pt plus 2pt minus 1pt}}
}{% if KOMA class
  \KOMAoptions{parskip=half}}
\makeatother
\usepackage{xcolor}
\usepackage[margin=1in]{geometry}
\usepackage{longtable,booktabs,array}
\usepackage{calc} % for calculating minipage widths
% Correct order of tables after \paragraph or \subparagraph
\usepackage{etoolbox}
\makeatletter
\patchcmd\longtable{\par}{\if@noskipsec\mbox{}\fi\par}{}{}
\makeatother
% Allow footnotes in longtable head/foot
\IfFileExists{footnotehyper.sty}{\usepackage{footnotehyper}}{\usepackage{footnote}}
\makesavenoteenv{longtable}
\usepackage{graphicx}
\makeatletter
\def\maxwidth{\ifdim\Gin@nat@width>\linewidth\linewidth\else\Gin@nat@width\fi}
\def\maxheight{\ifdim\Gin@nat@height>\textheight\textheight\else\Gin@nat@height\fi}
\makeatother
% Scale images if necessary, so that they will not overflow the page
% margins by default, and it is still possible to overwrite the defaults
% using explicit options in \includegraphics[width, height, ...]{}
\setkeys{Gin}{width=\maxwidth,height=\maxheight,keepaspectratio}
% Set default figure placement to htbp
\makeatletter
\def\fps@figure{htbp}
\makeatother
\setlength{\emergencystretch}{3em} % prevent overfull lines
\providecommand{\tightlist}{%
  \setlength{\itemsep}{0pt}\setlength{\parskip}{0pt}}
\setcounter{secnumdepth}{-\maxdimen} % remove section numbering
\ifLuaTeX
  \usepackage{selnolig}  % disable illegal ligatures
\fi
\usepackage[]{natbib}
\bibliographystyle{plainnat}
\IfFileExists{bookmark.sty}{\usepackage{bookmark}}{\usepackage{hyperref}}
\IfFileExists{xurl.sty}{\usepackage{xurl}}{} % add URL line breaks if available
\urlstyle{same} % disable monospaced font for URLs
\hypersetup{
  pdftitle={Democracy's Data: Analytics and Insights into American Elections},
  pdfauthor={Prof.~Jonathan Cervas},
  hidelinks,
  pdfcreator={LaTeX via pandoc}}

\title{Democracy's Data: Analytics and Insights into American Elections}
\usepackage{etoolbox}
\makeatletter
\providecommand{\subtitle}[1]{% add subtitle to \maketitle
  \apptocmd{\@title}{\par {\large #1 \par}}{}{}
}
\makeatother
\subtitle{{[}84-355{]} \textbar{} Fall 2025}
\author{Prof.~Jonathan Cervas}
\date{Updated: November 11, 2025}

\begin{document}
\maketitle

\begin{quote}
Professor Jonathan Cervas\\
Office: Posner Hall 374\\
Email:
\textbf{\href{mailto:cervas@cmu.edu}{\nolinkurl{cervas@cmu.edu}}}\\
Location: WEH 4623\\
Time: Tuesday 11:00a-12:20p Eastern\\
Office Hours: Wednesdays 2-4p \& by appointment\\
\href{https://www.cmu.edu/hub/calendar/}{\textbf{CMU Academic Calendar}}
\end{quote}

The most up-to-date version of this
\href{https://github.com/jcervas/teaching/tree/main/2025-2026/class-cmu-2025-84-355}{\textbf{syllabus
can be found here}:
https://github.com/jcervas/teaching/tree/main/2025-2026/class-cmu-2025-84-355}

\begin{center}\rule{0.5\linewidth}{0.5pt}\end{center}

\begin{quote}
\textbf{Prerequisites:} 36-200 Reasoning with Data\\
\textbf{Course Relevance:} DC: \emph{Perspectives on Justice and
Injustice}
\end{quote}

\begin{center}\rule{0.5\linewidth}{0.5pt}\end{center}

\hypertarget{course-description}{%
\subsection{Course Description:}\label{course-description}}

American democracy is rich with data --- from historical vote tallies to
modern polling, turnout, and campaign finance. In this course, we'll
investigate how democracy functions by analyzing this data, uncovering
the political, social, and structural forces that shape electoral
outcomes. Students will engage with historical case studies (e.g., 1876,
1960, 2000) alongside contemporary elections to see how past events
illuminate present dynamics. Through lectures, labs, and projects,
students will gain the tools to collect, analyze, and interpret
electoral data --- and apply these skills to real-world political
questions.

\hypertarget{course-goals}{%
\subsection{Course Goals}\label{course-goals}}

By the end of this course, students will:

\begin{itemize}
\tightlist
\item
  Understand how data can illuminate the functioning of democratic
  systems.
\item
  Identify historical and contemporary forces that shape electoral
  outcomes.\\
\item
  Feel comfortable looking at, wrangling, and interpreting data\\
\item
  Connect patterns in data to broader questions about political power,
  representation, and change.
\end{itemize}

\hypertarget{learning-objectives}{%
\subsection{Learning Objectives}\label{learning-objectives}}

Students will be able to:

\begin{enumerate}
\def\labelenumi{\arabic{enumi}.}
\tightlist
\item
  Collect and clean real-world electoral datasets.
\item
  Conduct (basic) statistical analyses to identify trends in voting
  behavior.\\
\item
  Interpret and visualize election data for academic and public
  audiences.
\item
  Evaluate the impact of historical events on present electoral
  dynamics.\\
\item
  Critically assess the strengths and limitations of various data
  sources.
\end{enumerate}

\hypertarget{assessments}{%
\subsection{Assessments}\label{assessments}}

\begin{itemize}
\tightlist
\item
  \textbf{Participation \& Attendance} -- 20\%\\
  Active engagement in lectures and labs.
\item
  \textbf{Weekly Lab Assignments} -- 50\%\\
  Hands-on data analysis exercises using real election datasets.
\item
  \textbf{Data Journalism Project} -- 10\%\\
  Produce a data-driven news story that investigates a significant
  trend, pattern, or issue in electoral processes.
\item
  \textbf{Policy Proposal or Data Journalism Project} -- 10\%\\
  Develop a policy proposal that addresses a specific issue in the
  electoral process.
\item
  \textbf{Weekly Data Contributions} -- 10\%\\
  Submit one relevant data point related to politics or democracy,
  accompanied by a brief explanation and visualization to illustrate its
  significance.
\end{itemize}

\hypertarget{topics}{%
\subsection{Topics}\label{topics}}

\begin{enumerate}
\def\labelenumi{\arabic{enumi}.}
\tightlist
\item
  \textbf{Introduction} -- Why study democracy with data?\\
\item
  \textbf{Elections as Data} -- Sources, structures, and pitfalls.\\
\item
  \textbf{Population \& Demographics} -- How population shifts shape
  politics.\\
\item
  \textbf{Racial Threat Theory vs.~Contact Theory} -- How racial
  dynamics shape voting and policy.\\
\item
  \textbf{Racial Resentment vs.~Principled Conservatism} -- Prejudice or
  ideology?\\
\item
  \textbf{Public Opinion \& Survey Data} -- Using the ANES and other
  sources to understand attitudes.\\
\item
  \textbf{Role of Partisan Identity} -- How party affiliation influences
  voter behavior.\\
\item
  \textbf{Malapportionment and Formal Representation} -- The impact of
  districting and institutions on representation.\\
\item
  \textbf{Electoral College} -- How it works, its effects, and
  alternatives.\\
\item
  \textbf{Campaign Strategies} -- Messaging, media, and targeting.\\
\item
  \textbf{Redistricting \& Gerrymandering} -- The geometry of
  representation and its impact.\\
\item
  \textbf{Voter Turnout} -- What drives participation? Which parties
  benefit from higher turnout?\\
\item
  \textbf{Rational Voter vs.~Low-Information Voter} -- Are voters
  policy-driven or guided by heuristics?\\
\item
  \textbf{Economic vs.~Cultural Polarization} -- Are partisan divides
  driven by economics or cultural identity?\\
\item
  \textbf{Political Polarization} -- Causes and consequences for
  governance.\\
\item
  \textbf{Polling \& Forecasting} -- Methods, models, and uncertainty.\\
\item
  \textbf{Media Effects} -- Does media reinforce views or change
  minds?\\
\item
  \textbf{Partisan Media} -- The rise of partisan news sources and their
  effects.\\
\item
  \textbf{Misinformation \& Trust in Elections} -- Data and public
  opinion.\\
\item
  \textbf{Campaign Finance} -- Who funds elections and why it matters.\\
\item
  \textbf{The Parties in Our Heads} -- How partisanship shapes our views
  of the world.\\
\item
  \textbf{Election Administration} -- Laws, technology, and integrity.\\
\item
  \textbf{Median Voter Theorem vs.~Party Polarization} -- Do parties
  converge or cater to their bases?\\
\item
  \textbf{Economic Inequality's Impact on Democracy} -- Does inequality
  weaken democracy?
\end{enumerate}

\hypertarget{assignments}{%
\subsection{Assignments}\label{assignments}}

\begin{enumerate}
\def\labelenumi{\arabic{enumi}.}
\item
  \textbf{Current Events Data Snippets:}

  \begin{itemize}
  \tightlist
  \item
    You must submit one piece of relevant data on the Canvas discussion
    board (slide on link provided) before class on Tuesday.
  \item
    You can access free articles and archives from the New York Times
    and other major newspapers using your \texttt{cmu.edu} email.
  \item
    You may be called on randomly to share your data nugget.
  \item
    I will create a shared powerpoint where you can post your slide.
  \item
    They will be graded on a completion basis (complete/incomplete).
  \end{itemize}

  \begin{quote}
  \textbf{Example:}\\
  \emph{Voter Turnout}:
  \texttt{About\ two-thirds\ (66\%)\ of\ the\ voting-eligible\ population\ turned\ out\ for\ the\ 2020\ presidential\ election}
  \footnote{\textbf{Pew Research Center}}(\url{https://www.pewresearch.org/politics/2023/07/12/voter-turnout-2018-2022/}).\\
  Create info-graphic visualizations to illustrate the data, to be
  posted on common slideshow. Link is provided on Canvas
  \end{quote}
\item
  \textbf{Weekly Lab Assignments}

  \begin{itemize}
  \tightlist
  \item
    Each week, you will complete a lab assignment that involves
    analyzing a dataset related to the week`s topic.
  \item
    These assignments will help you practice data cleaning, analysis,
    and visualization techniques.
  \item
    The assignments will be graded on a completion basis
    (complete/incomplete).
  \end{itemize}
\item
  \textbf{Data Journalism Project:} This project requires you to write a
  compelling, data-driven news story investigating a significant
  electoral trend, pattern, or issue. You'll combine original data
  analysis, at least two visualizations, expert interviews, and
  historical context to create an engaging narrative that explains how
  and why democracy functions as it does.
\item
  \textbf{Policy Proposals:}\\
  This assignment asks you to develop a clear, evidence-based policy
  proposal addressing a specific electoral process issue such as voter
  turnout, election security, or poll accuracy. You'll use data,
  historical analysis, and real-world examples to identify the problem,
  propose a solution, and justify its effectiveness for policymakers.
\item
  \textbf{Attendance:}\\
  Regular attendance and active involvement form a significant part of
  your final grade (see grading section). If you do not show up, you
  will not earn an `A'. Participation is not just about being present;
  it involves engaging with the material, contributing to discussions,
  and collaborating with your peers. To recognize that occasional
  absences are sometimes unavoidable (e.g., for religious observance,
  job interviews, university-sanctioned events, or illness), attendance
  grades will be calculated using an exponential function. 1--2 absences
  → mild penalty, 6+ absences → sharp drop (serious consequences).
\end{enumerate}

\begin{figure}
\centering
\includegraphics{/Users/cervas/Library/CloudStorage/GoogleDrive-jcervas@andrew.cmu.edu/My Drive/GitHub/teaching/2025-2026/class-cmu-2025-84-355/readme_files/figure-latex/plot-curve-1.png}
\caption{Effect of Absences on Grade}
\end{figure}

Students are expected and encouraged to meet all deadlines for
assignments. If you are unable to complete the assignment work by the
due date, reach out in advance to make alternative arrangements. I
typically will not penalize you for turning in your assignment late, so
long as it does not hinder completion of other`s work (ie, group
projects).

\begin{center}\rule{0.5\linewidth}{0.5pt}\end{center}

The course grade will be a weighted average of the following components:

\begin{longtable}[]{@{}ll@{}}
\toprule()
Category & Percent of Final Grade \\
\midrule()
\endhead
\textbf{Participation \& Attendance} & 20\% \\
\textbf{Weekly Lab Assignments} & 50\% \\
\textbf{Midterm Project} & 10\% \\
\textbf{Policy Proposal} & 10\% \\
\textbf{Weekly Data Contributions} & 10\% \\
\bottomrule()
\end{longtable}

\begin{center}\rule{0.5\linewidth}{0.5pt}\end{center}

\hypertarget{schedule}{%
\subsection{Schedule}\label{schedule}}

\textbf{Week 1}

\begin{itemize}
\tightlist
\item
  \textbf{Labs}:

  \begin{itemize}
  \tightlist
  \item
    Two Truths and a Lie\\
  \item
    2024 Electoral Map\\
  \item
    ANES Scavenger Hunt
  \end{itemize}
\end{itemize}

\textbf{Week 2}

\begin{itemize}
\tightlist
\item
  Bailey, Michael A. 2024. Polling at a Crossroads: Rethinking Modern
  Survey Research. Cambridge: Cambridge University Press.
  \textbf{p.~3-22}

  \begin{itemize}
  \tightlist
  \item
    \url{https://canvas.cmu.edu/courses/47828/files?preview=12869135Pages}
  \end{itemize}
\item
  \textbf{Optional Reading}: Herbert F. Weisberg, Reflections: The
  Michigan Four and Their Study of American Voters: A Biography of a
  Collaboration, 49 PS: Political Science \& Politics 845 (2016)

  \begin{itemize}
  \tightlist
  \item
    \url{https://www.cambridge.org/core/journals/ps-political-science-and-politics/article/reflections-the-michigan-four-and-their-study-of-american-voters-a-biography-of-a-collaboration/77EFAE6D9A6A0F3532922C720F7BE0E4}.
  \end{itemize}
\item
  \textbf{Labs}:

  \begin{itemize}
  \tightlist
  \item
    Sensitive Interview Questions
  \item
    Questionaire Writing
  \end{itemize}
\end{itemize}

\textbf{Week 3}

\begin{itemize}
\tightlist
\item
  NCSL Redistricting Law 2020, Chapter 1, The Census

  \begin{itemize}
  \tightlist
  \item
    \url{https://canvas.cmu.edu/files/13154611/download?download_frd=1\&verifier=H7yO9sICuYEpjpM6t01ze39jU5rSxZ6rDp5tA9fS}
  \end{itemize}
\end{itemize}

\textbf{Week 4}

\begin{itemize}
\tightlist
\item
  Introductory to the American Community Survey (ACS) {[}Video{]}

  \begin{itemize}
  \tightlist
  \item
    \url{https://youtu.be/eqKi1l2F4xQ?si=Hoh1m2kUOMtlBVCP}
  \end{itemize}
\end{itemize}

\textbf{Week 5}

\begin{itemize}
\tightlist
\item
  No New Readings
\end{itemize}

\textbf{Week 6}

\begin{itemize}
\tightlist
\item
  Surname Analysis

  \begin{itemize}
  \item
    \begin{enumerate}
    \def\labelenumi{\arabic{enumi}.}
    \tightlist
    \item
      Kosuke Imai \& Kabir Khanna,
      \href{https://canvas.cmu.edu/courses/47828/files/13281633?wrap=1}{Improving
      Ecological Inference by Predicting Individual Ethnicity from Voter
      Registration Records}, 24 Polit. anal. 263 (2016)
    \end{enumerate}
  \end{itemize}
\end{itemize}

\textbf{Week 7}

\begin{itemize}
\tightlist
\item
  Guest Speaker, \textbf{Jonathan Lai}

  \begin{itemize}
  \tightlist
  \item
    Pitch topics for Data Journalism
  \end{itemize}
\end{itemize}

\textbf{Week 8}

\begin{itemize}
\tightlist
\item
  Surname Analysis

  \begin{itemize}
  \tightlist
  \item
    Estimation and Coding
  \end{itemize}
\end{itemize}

\textbf{Week 9}

\begin{itemize}
\tightlist
\item
  Redistricting and Gerrymandering

  \begin{itemize}
  \tightlist
  \item
    NCSL Book, Chapter 4 (Redistricting Principles and Criteria)\\
  \item
    Is drawing a voting map that helps a political party illegal? Only
    in some states:
    \url{https://www.npr.org/2023/05/17/1173469584/partisan-gerrymandering-explainer-north-carolina}\\
  \item
    Gerrymandering Explained:
    \url{https://www.brennancenter.org/our-work/research-reports/gerrymandering-explained}
  \end{itemize}
\end{itemize}

Sign up for DRA: \url{https://davesredistricting.org/maps\#home} Links
to an external site.

\textbf{Week 10}

\begin{itemize}
\tightlist
\item
  Parisan Gerrymandering

  \begin{itemize}
  \tightlist
  \item
    \textbf{Chapter 16} - Redistricing Algorithms - Political Geometry:
    Rethinking Redistricting in the US with Math, Law, and Everything In
    Between (Moon Duchin \& Olivia Walch eds., 2022),
    \url{https://link.springer.com/10.1007/978-3-319-69161-9}.
  \end{itemize}
\end{itemize}

\textbf{Week 11}

\begin{itemize}
\tightlist
\item
  Racial Polarization and Redistricting

  \begin{itemize}
  \item
    Racially Polarized Voting, Redistricting Data Hub,
    \url{https://redistrictingdatahub.org/resources/racially-polarized-voting/}
  \item
    John Sides \& Michael Tesler, U.S. Voters Are Increasingly Polarized
    about Race, the Data Show., Good Authority,
    \url{https://goodauthority.org/news/america-is-less-polarized-by-race-but-more-polarized-about-race/}
  \item
    Nadia E. Brown \& Michael G. Strawbridge, A Key Part of the 1965
    Voting Rights Act Is under Attack, Good Authority,
    \url{https://goodauthority.org/news/a-key-part-of-the-1965-voting-rights-act-is-under-attack/}
  \item
    A Solution to the Ecological Inference Problem: reconstructing
    individual behavior from aggregate data -
    \href{https://gking.harvard.edu/sites/g/files/omnuum7116/files/gking/files/part1.pdf}{Chapter
    1}
  \end{itemize}
\end{itemize}

\hypertarget{ai-use-policy-for-student-work}{%
\subsection{AI Use Policy for Student
Work}\label{ai-use-policy-for-student-work}}

As artificial intelligence (AI) tools become increasingly accessible, it
is important to clarify expectations for their use in this course. You
are welcome to use AI technologies (such as ChatGPT, Grammarly, or
similar tools) to support your independent work---such as brainstorming
ideas, checking grammar, or improving the clarity of your writing.
However, you \textbf{may not use AI to generate substantive content that
you submit as your own original work}. All assignments, essays, and
projects must reflect your own analysis, critical thinking, and voice.

\textbf{Permitted Uses of AI:}

\begin{itemize}
\tightlist
\item
  Outlining or organizing your thoughts
\item
  Checking grammar, spelling, or clarity
\item
  Generating ideas or prompts to help you get started
\item
  Reviewing your own drafts for readability
\end{itemize}

\textbf{Prohibited Uses of AI:}

\begin{itemize}
\tightlist
\item
  Submitting AI-generated essays, paragraphs, or answers as your own
  work
\item
  Using AI to complete assignments, discussion posts, or projects in
  place of your own effort
\item
  Copying and pasting AI-generated content without substantial revision
  and personal input
\end{itemize}

If you use AI tools in your process, you must \textbf{disclose} how you
used them in a brief note at the end of your assignment (e.g., ``I used
ChatGPT to help brainstorm ideas for my outline.'').

\textbf{Violations:}\\
Submitting AI-generated content as your own is considered academic
dishonesty and will be treated as a violation of the university's
academic integrity policy.

If you have questions about what is or is not allowed, please ask before
submitting your work.

\hypertarget{representation-statement}{%
\subsection{Representation Statement}\label{representation-statement}}

I am committed to including a broad range of perspectives in the
readings and materials for this course. If you believe a critical voice
is missing, please let me know so I can improve the syllabus now and in
future offerings.

\textbf{We must treat every individual with respect.} We come from many
different backgrounds, and this variety of viewpoints is fundamental to
building and maintaining an equitable and inclusive campus community.
``Representation'' can refer to the ways we identify ourselves---race,
color, national origin, language, sex, disability, age, sexual
orientation, gender identity, religion, creed, ancestry, belief, veteran
status, or genetic information, among others. Each of these identities
shapes the perspectives our students, faculty, and staff bring to
campus. Promoting these varied viewpoints not only fuels excellence and
innovation but also advances the pursuit of justice. We acknowledge our
imperfections while fully committing to the work---inside and outside
our classrooms---of building and sustaining a campus community that
embraces these core values.

Each of us is responsible for creating a safer, more inclusive
environment.

Unfortunately, incidents of bias or discrimination do occur, whether
intentional or unintentional. They contribute to an unwelcoming
atmosphere for individuals and groups at the university. Therefore, the
university encourages anyone who experiences or observes unfair or
hostile treatment on the basis of identity to speak out for justice and
seek support---either in the moment or afterward. You can share your
experiences using the following resources:

\begin{itemize}
\tightlist
\item
  \textbf{Ethics Reporting Hotline}\\
  Submit an anonymous report by calling 844-587-0793 or visiting
  \textbf{cmu.ethicspoint.com}.\\
  All reports are documented and reviewed to determine whether further
  action is needed. Regardless of the incident type, the university will
  use your feedback to transform our campus climate into one that is
  more equitable and just.
\end{itemize}

\hypertarget{accommodations-for-students-with-disabilities}{%
\subsection{Accommodations for Students with
Disabilities}\label{accommodations-for-students-with-disabilities}}

If you have a documented disability and an accommodations letter from
the Office of Disability Resources, please discuss your needs with me as
early in the semester as possible. I will work with you to ensure that
accommodations are provided as appropriate. If you suspect you may have
a disability and are not yet registered with the Office of Disability
Resources, you can contact them at
\textbf{\href{mailto:access@andrew.cmu.edu}{\nolinkurl{access@andrew.cmu.edu}}}.

\hypertarget{student-well-being}{%
\subsection{Student Well-Being}\label{student-well-being}}

The past few years have been challenging. We are all under significant
stress and uncertainty. I encourage you to find ways to move regularly,
eat well, and reach out to your support system---or to me at
\textbf{\href{mailto:cervas@cmu.edu}{\nolinkurl{cervas@cmu.edu}}}---if
you need help. We can all benefit from support during stressful times,
and this semester is no exception.

As a student, you may experience a range of challenges that interfere
with learning, such as strained relationships, increased anxiety,
substance use, feeling down, difficulty concentrating, or lack of
motivation. These mental health concerns or stressful events can
diminish your academic performance and reduce your ability to
participate in daily activities. CMU offers services that can help, and
treatment does work. Learn more about confidential mental health
services available on campus at:

\begin{itemize}
\tightlist
\item
  \textbf{Counseling and Psychological Services}:
  \url{http://www.cmu.edu/counseling/}\\
  Phone (24/7): 412-268-2922
\end{itemize}

Please remember that support is always available---don't hesitate to
reach out.

\begin{center}\rule{0.5\linewidth}{0.5pt}\end{center}

\hypertarget{major-debates-in-american-politics-reference-sheet}{%
\section{Major Debates in American Politics: Reference
Sheet}\label{major-debates-in-american-politics-reference-sheet}}

\begin{center}\rule{0.5\linewidth}{0.5pt}\end{center}

\hypertarget{public-opinion-political-behavior}{%
\subsection{1. Public Opinion \& Political
Behavior}\label{public-opinion-political-behavior}}

\begin{itemize}
\tightlist
\item
  \textbf{Racial Threat Theory vs.~Contact Theory}

  \begin{itemize}
  \tightlist
  \item
    \emph{Key Question:} Does diversity increase prejudice or reduce it
    through interaction?\\
  \item
    \emph{Representative Scholars/Studies:} V.O. Key (1949); Blalock
    (1967); Allport (1954); Pettigrew \& Tropp (2006)
  \end{itemize}
\item
  \textbf{Economic vs.~Cultural Polarization}

  \begin{itemize}
  \tightlist
  \item
    \emph{Key Question:} Are partisan divides driven by economics or
    cultural identity?\\
  \item
    \emph{Representative Scholars/Studies:} Hochschild (2016); Inglehart
    \& Norris (2016); Autor, Dorn, Hanson (2013)
  \end{itemize}
\item
  \textbf{Rational Voter vs.~Low-Information Voter}

  \begin{itemize}
  \tightlist
  \item
    \emph{Key Question:} Do voters make decisions based on policy or
    heuristics?\\
  \item
    \emph{Representative Scholars/Studies:} Downs (1957); Converse
    (1964)
  \end{itemize}
\item
  \textbf{Partisan Identity as Social Identity}

  \begin{itemize}
  \tightlist
  \item
    \emph{Key Question:} Is partisanship a social identity or rational
    policy choice?\\
  \item
    \emph{Representative Scholars/Studies:} Campbell et al.~(1960);
    Green, Palmquist, \& Schickler (2002)
  \end{itemize}
\end{itemize}

\begin{center}\rule{0.5\linewidth}{0.5pt}\end{center}

\hypertarget{institutions-representation}{%
\subsection{2. Institutions \&
Representation}\label{institutions-representation}}

\begin{itemize}
\tightlist
\item
  \textbf{Majoritarianism vs.~Countermajoritarianism}

  \begin{itemize}
  \tightlist
  \item
    \emph{Key Question:} Should institutions reflect majority will or
    protect minorities?\\
  \item
    \emph{Representative Scholars/Studies:} Dahl (1956); Bickel (1962)
  \end{itemize}
\item
  \textbf{Descriptive vs.~Substantive Representation}

  \begin{itemize}
  \tightlist
  \item
    \emph{Key Question:} Does shared identity between reps and
    constituents matter?\\
  \item
    \emph{Representative Scholars/Studies:} Pitkin (1967); Mansbridge
    (1999)
  \end{itemize}
\item
  \textbf{Electoral College \& Malapportionment}

  \begin{itemize}
  \tightlist
  \item
    \emph{Key Question:} Do these features protect federalism or
    undermine equality?\\
  \item
    \emph{Representative Scholars/Studies:} Edwards (2004); Lee \&
    Oppenheimer (1999)
  \end{itemize}
\end{itemize}

\begin{center}\rule{0.5\linewidth}{0.5pt}\end{center}

\hypertarget{federalism-state-power}{%
\subsection{3. Federalism \& State Power}\label{federalism-state-power}}

\begin{itemize}
\tightlist
\item
  \textbf{Centralization vs.~Decentralization}

  \begin{itemize}
  \tightlist
  \item
    \emph{Key Question:} Should policy be set nationally or locally?\\
  \item
    \emph{Representative Scholars/Studies:} Riker (1964); Kincaid (1990)
  \end{itemize}
\item
  \textbf{Policy Diffusion}

  \begin{itemize}
  \tightlist
  \item
    \emph{Key Question:} Do states innovate and spread good policy or
    reinforce inequality?\\
  \item
    \emph{Representative Scholars/Studies:} Walker (1969); Berry \&
    Berry (1990)
  \end{itemize}
\end{itemize}

\begin{center}\rule{0.5\linewidth}{0.5pt}\end{center}

\hypertarget{political-polarization}{%
\subsection{4. Political Polarization}\label{political-polarization}}

\begin{itemize}
\tightlist
\item
  \textbf{Elite-Driven vs.~Mass-Driven Polarization}

  \begin{itemize}
  \tightlist
  \item
    \emph{Key Question:} Are elites or the public the primary driver of
    polarization?\\
  \item
    \emph{Representative Scholars/Studies:} Fiorina et al.~(2005);
    Abramowitz \& Saunders (2008)
  \end{itemize}
\item
  \textbf{Asymmetric Polarization}

  \begin{itemize}
  \tightlist
  \item
    \emph{Key Question:} Is polarization equal on both sides or
    skewed?\\
  \item
    \emph{Representative Scholars/Studies:} McCarty, Poole, \& Rosenthal
    (2006); Mann \& Ornstein (2012)
  \end{itemize}
\end{itemize}

\begin{center}\rule{0.5\linewidth}{0.5pt}\end{center}

\hypertarget{race-ethnicity-politics}{%
\subsection{5. Race, Ethnicity, \&
Politics}\label{race-ethnicity-politics}}

\begin{itemize}
\tightlist
\item
  \textbf{Linked Fate vs.~Individualism}

  \begin{itemize}
  \tightlist
  \item
    \emph{Key Question:} Do marginalized groups vote as a bloc due to
    shared fate?\\
  \item
    \emph{Representative Scholars/Studies:} Dawson (1994); McClain et
    al.~(2009)
  \end{itemize}
\item
  \textbf{Racial Resentment vs.~Principled Conservatism}

  \begin{itemize}
  \tightlist
  \item
    \emph{Key Question:} Is opposition to minority-focused policy driven
    by prejudice or ideology?\\
  \item
    \emph{Representative Scholars/Studies:} Kinder \& Sears (1981);
    Kinder \& Kam (2009); Sniderman \& Carmines (1997)
  \end{itemize}
\end{itemize}

\begin{center}\rule{0.5\linewidth}{0.5pt}\end{center}

\hypertarget{political-economy}{%
\subsection{6. Political Economy}\label{political-economy}}

\begin{itemize}
\tightlist
\item
  \textbf{Median Voter Theorem vs.~Party Polarization}

  \begin{itemize}
  \tightlist
  \item
    \emph{Key Question:} Do parties converge to the median voter or
    cater to their bases?\\
  \item
    \emph{Representative Scholars/Studies:} Downs (1957); Ansolabehere
    et al.~(2001)
  \end{itemize}
\item
  \textbf{Economic Inequality \& Democracy}

  \begin{itemize}
  \tightlist
  \item
    \emph{Key Question:} Does inequality weaken democracy or can
    institutions buffer it?\\
  \item
    \emph{Representative Scholars/Studies:} Gilens (2012); Bartels
    (2008)
  \end{itemize}
\end{itemize}

\begin{center}\rule{0.5\linewidth}{0.5pt}\end{center}

\hypertarget{political-communication}{%
\subsection{7. Political Communication}\label{political-communication}}

\begin{itemize}
\tightlist
\item
  \textbf{Media Effects: Minimal vs.~Strong}

  \begin{itemize}
  \tightlist
  \item
    \emph{Key Question:} Does media mainly reinforce views or change
    minds?\\
  \item
    \emph{Representative Scholars/Studies:} Lazarsfeld et al.~(1944);
    Iyengar \& Kinder (1987)
  \end{itemize}
\item
  \textbf{Social Media \& Polarization}

  \begin{itemize}
  \tightlist
  \item
    \emph{Key Question:} Is social media polarizing politics or
    reflecting existing divides?\\
  \item
    \emph{Representative Scholars/Studies:} Bail et al.~(2018); Tucker
    et al.~(2018)
  \end{itemize}
\end{itemize}

  \bibliography{bib.bib}

\end{document}
